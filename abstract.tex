
\begin{abstract}
 
The Vera C. Rubin Observatory is an 8m-class optical facility currently under construction on Cherro Pachón, Chile.  During the 10-year Legacy Survey of Space and Time (LSST) starting in 2022 the Vera C. Rubin Observatory will scan the entire accessible night sky roughly every 3 days. A special observational cadence allows for the discovery and characterization of millions of Solar System Objects, among them roughly two thirds of all near-Earth Objects (NEOs) with absolute magnitude H < 22 mag. Data products encompass a near real time alert stream to enable rapid follow-up as well as orbit catalogs exclusively based on high precision astrometry published on a yearly basis. The latter are valuable input for population debiasing and future survey simulation studies. Here, we present an overview of the expected Solar System Processing capabilities during the LSST.
\end{abstract}

